% The font could be set to Windows-specific Calibri by using the 'calibri' option
\documentclass[]{fraguilarcv}

% For brazilian accentuation
\usepackage[brazilian]{babel}
\usepackage[utf8]{inputenc}
\usepackage[T1]{fontenc}
% For mathematical symbols
\usepackage{amsmath}

% Header Personal Data
\name {Fernando R. Aguilar}
\address {Sobradinho, Brasília, \linebreak Distrito Federal, Brasil}
\contacts {(61) 99906-0320 \linebreak fernando@aguilar.net.br \linebreak @fernand0aguilar}

\begin{document}

	% Print the header
	\makeheader

	% Print the content of employments
	\begin{cvsection}{Experiência}
    \begin{cvsubsection}{Estagiário}{Medialab/UnB}{(05/2017 - Atualmente)}
			\begin{itemize}
              \item Estou fazendo o design e desenvolvendo o website do laboratório.
              \item Laboratório de midias digitais da UnB. Segue o padrão de interseção entre arte e tecnologia.
              \item Aprendi o conceito de poética, entrei em contato com novas mentalidades criativas.
			\end{itemize}
		\end{cvsubsection}
		\begin{cvsubsection}{Desenvolvedor, Estagiário}{Agência Espacial Brasileira}{(01/2017 - 05/2017)}
			\begin{itemize}
              \item Responsável pela manutenção e solução de problemas do sistema web de recursos humanos.
              \item Evolui minha organização, habilidades orientadas a solução de problemas e capacidade de trabalhar sozinho depois de dadas as instruções.
              \item Sugeri mudanças e melhorias na metodologia. Grande contato com a cultura do design UX e DevOps.
			\end{itemize}
		\end{cvsubsection}

		\begin{cvsubsection}{Desenvolvedor}{Sistema de Monitoramento Energético - UnB}{(06/2016 - 12/2016)}
			\begin{itemize}
              \item Concepção de um sistema web para realizar o monitoramento instantâneo do consumo de energia elétrica em predios da Universidade.
              \item Compreendi a importância de um processo bem implementado.
              \item Aprendi a ser mais colaborativo e a motivar a equipe em momentos de dificuldades.
              \item Resolvi constantes problemas de configuração aplicando conceitos DevOps.
			\end{itemize}
		\end{cvsubsection}

		\begin{cvsubsection}{Desenvolvedor, Estagiário}{UeBrasil - Mahvla}{(12/2015 - 06/2016)}
			\begin{itemize}
				\item Evolui o sistema de monitoramento das Tornozeleiras Eletrônicas de detentos dos estados: RS e BA.
	      		\item Com o contexto de sistema crítico, refatorei funcionalidades essenciais do projeto, reduzindo a complexibilidade e auxiliando a escalabilidade de novo contrato.
	      		\item Participei na análise dos Requisitos, fazendo protótipos e histórias de usuários;
			\end{itemize}
		\end{cvsubsection}

	%Print the content of eductations%
	\begin{cvsection}{Educação}
		\begin{cvsubsection}{Gama, Brasilia, DF}{Universidade de Brasília (UnB)}{(06/2014 - Atualmente)}
			\begin{itemize}
				\item Graduação em Engenharia de Software
				\item \textbf{Coursework:} {Processos de Desenvolvimento de Software; Gerenciamento de Configuração; Gestão da Produção e Qualidade; Sistemas Operacionais; Fundamentos de Redes de Computadores;}
			\end{itemize}
		\end{cvsubsection}
	\end{cvsection}

	%Print the content of technical Experience%
	\begin{cvsection}{Experiência Técnica}
		\begin{cvsubsection}{Projetos:}{}{}
			\begin{itemize}
	    	\item \textbf{MaisMorra Game (2017):} Jogo finalista da Epic Game Jam. Vencedor da votação popular da região de Brasília, o que garantiu a vaga para a final que acontecerá no Rock in Rio 2017. Permitiu conhecimento de programação, liderança, marketing e business  [C\#, Unity].
			\item \textbf{CrossRoads Game (2017):} Construção da Engine e lógica de jogo. Programação, gerenciamento e interpretação de ideias. Liderando um time multi-funcional de 6 pessoas. [ C++, SDL 2 ]
			\item \textbf{Iniciação Científica (2017):} Modelagem e análise de conceitos das funções meta-fibonacci de Hofstadter, usando algoritmos genéticos. Orientador: Dr. Ronni Amorim.
	      	\item \textbf{Portal Solidaridad (2016):} Plataforma de Comércio eletrônico que objetiva dar mais liberdade financeira a algumas entidades beneficentes de Brasília (solidaridad.com.br)
            \item \textbf{Estação Meteorológica (2015):} Desenvolvimento de um sistema para medição de dados do tempo. Interfaceamento em Arduino e criação de um aplicativo para Android.
			\end{itemize}
		\end{cvsubsection}
	\end{cvsection}
\end{cvsection}
\end{document}
