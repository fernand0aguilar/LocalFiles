% The font could be set to Windows-specific Calibri by using the 'calibri' option
\documentclass[]{fraguilarcv}

% For brazilian accentuation
\usepackage[brazilian]{babel}
\usepackage[utf8]{inputenc}
\usepackage[T1]{fontenc}
% For mathematical symbols
\usepackage{amsmath}

% Header Personal Data
\name {Fernando R. Aguilar}
\address {Grande Colorado, Brasília, \linebreak 20 anos.}
\contacts {(61) 999.060.320 \linebreak fernando@aguilar.net.br \linebreak @fraguilar}

\begin{document}

	% Print the header
	\makeheader

	% Print the content of employments
	\begin{cvsection}{Experiência}
    \begin{cvsubsection}{Estagiário}{Medialab/UnB}{(06/2017 - Atualmente)}
			\begin{itemize}
              \item \textbf{Descrição:} Laboratório multidisciplinar de Arte e Tecnologia ou Arte Computacional.
              \item \textbf{Atividades:} Reconhecimento da função do artista na sociedade. Foi monitor na exposição (\#16 ART: Artis intelligentia) e possui coautoria nas obras (Ciberflor, medialab.unb.br, projeção fractal-dome, Ipê Amarelo). 
              \item \textbf{Aprendizado:} O valor de colocar em movimento o potencial criativo, experienciando a evolução pessoal através da presença consciente nas práticas e reflexões cotidianas.
			\end{itemize}
		\end{cvsubsection}
		\begin{cvsubsection}{Estagiário}{Agência Espacial Brasileira}{(01/2017 - 05/2017)}
			\begin{itemize}
              \item \textbf{Descrição:} Divisão responsável pela manutenção e evolução dos sistemas tecnológicos da organização que coordena a política espacial brasileira.
              \item \textbf{Atividades:} Encarregado de atender chamados dos sistemas web - AEBRH (recursos humanos) e IMATEC (índice de maturidade tecnológica) - Sugeriu melhorias na metodologia e no desenho grupal.
              \item \textbf{Aprendizado:} Evoluiu a organização, habilidade de solucionar problemas complexos e a capacidade de trabalhar sozinho depois de dadas as instruções. Entrou em contato com a cultura do design voltado ao usuário e ao desenvolvimento integrado com as operações e estrutura.
			\end{itemize}
		\end{cvsubsection}

		\begin{cvsubsection}{Desenvolvedor}{Sistema de Monitoramento Energético - UnB}{(06/2016 - 12/2016)}
			\begin{itemize}
              \item \textbf{Descrição:} Concepção de um sistema web para monitorar em tempo real o consumo de energia elétrica em prédios da Universidade.             
              \item \textbf{Atividades:} Além de desenvolver funcionalidades do sistema, solucionou problemas constantes da equipe aplicando padrões de gerência de configuração de software.
              \item \textbf{Aprendizado:} A importância e o valor da mediação em ambientes grupais.
			\end{itemize}
		\end{cvsubsection}

		\begin{cvsubsection}{Estagiário}{UeBrasil - Grupo Mahvla}{(12/2015 - 06/2016)}
			\begin{itemize}
				\item \textbf{Descrição:} Sistema de controle e monitoramento das tornozeleiras eletrônicas para detentos
	      		\item \textbf{Atividades:} Com o contexto de sistema crítico, participei evoluindo e refatorando funcionalidades essenciais, reduzindo a complexibilidade e auxiliando a escalabilidade de novo contrato no governo do DF.
                \item \textbf{Aprendizado:} Mentorado pelo arquiteto de software, participei e obtive uma visão macro a respeito do processo, analisando requisitos, fazendo protótipos e desenvolvendo histórias de usuários;
			\end{itemize}
		\end{cvsubsection}

	%Print the content of eductations%
	\begin{cvsection}{Educação}
		\begin{cvsubsection}{Gama, Brasilia, DF}{Universidade de Brasília (UnB)}{(06/2014 - Atualmente)}
			\begin{itemize}
				\item Graduação em Engenharia de Software.
				\item \textbf{Coursework:} {Processos de Desenvolvimento de Software; Gerenciamento de Configuração; Gestão da Produção e Qualidade; Humanidades e Cidadania; Fundamentos de Redes de Computadores; ... ;}
			\end{itemize}
		\end{cvsubsection}
	\end{cvsection}

	%Print the content of technical Experience%
	\begin{cvsection}{Experiência Técnica}
		\begin{cvsubsection}{Projetos Pessoais:}{}{}
			\begin{itemize}
	    	\item \textbf{MaisMorra Game (2017):} Jogo finalista da Epic Game Jam, que ocorreu no Rock in Rio. Vencedor da votação popular de Brasília, permitiu conhecimentos de gestão, marketing digital, negócios, animação e programação.
	      	\item \textbf{Portal Solidaridad (2016):} Plataforma que objetiva garantir a liberdade financeira para as entidades beneficentes parceiras (solidaridad.com.br).
			\end{itemize}
		\end{cvsubsection}
	\end{cvsection}
\end{cvsection}
\end{document}
