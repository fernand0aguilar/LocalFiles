% The font could be set to Windows-specific Calibri by using the 'calibri' option
\documentclass[]{fraguilarcv}

% For brazilian accentuation
\usepackage[brazilian]{babel}
\usepackage[utf8]{inputenc}
\usepackage[T1]{fontenc}
% For mathematical symbols
\usepackage{amsmath}

% Header Personal Data
\name {Fernando R. Aguilar}
\address {Sobradinho, Brasília, \linebreak Distrito Federal, Brasil}
\contacts {(61) 99906-0320 \linebreak fernando@aguilar.net.br \linebreak @fernand0aguilar}

\begin{document}

	% Print the header
	\makeheader

	% Print the content of employments
	\begin{cvsection}{Experiência}
		\begin{cvsubsection}{Desenvolvedor FullStack, Estagiário}{Agência Espacial Brasileira}{(01/2017 - Atualmente)}
			\begin{itemize}
              \item
              \item
              \item
              \item
			\end{itemize}
		\end{cvsubsection}

		\begin{cvsubsection}{Back-End Developer}{Sistema Monitoramento Energético - UnB}{(06/2016 - 12/2016)}
			\begin{itemize}
              \item Concepção de um sistema web para realizar o monitoramento instantâneo de Interpretadores de Energia nas instalações da Universidade.
              \item Compreendi a importância de um processo bem implementado, bem como criei novos habitos focados em entrega e solução de problemas.
              \item Aprendi a ser mais colaborativo e a motivar a equipe em momentos de dificuldades.
              \item Resolvi constantes problemas de configuração aplicando conceitos DevOps.
			\end{itemize}
		\end{cvsubsection}

		\begin{cvsubsection}{Desenvolvedor, Estagiário}{UeBrasil - Mahvla}{(12/2015 - 06/2016)}
			\begin{itemize}
				\item Evolui o sistema de monitoramento das Tornozeleiras Eletrônicas de detentos dos estados RS, BA.
	      		\item Com o contexto de sistema crítico, refatorei funcionalidades essenciais do projeto, reduzindo a complexibilidade e escalando novo contrato - DF.
	      		\item Participei na análise dos Requisitos, sugerindo metas mensuráveis, fazendo protótipos, criando épicos e histórias de usuários;
			\end{itemize}
		\end{cvsubsection}

	%Print the content of eductations%
	\begin{cvsection}{Educação}
		\begin{cvsubsection}{Brasilia, DF}{Universidade de Brasília (UnB)}{(06/2014 - Present)}
			\begin{itemize}
				\item Graduação em Engenharia de Software. Minor em Engenharia Eletrônica.
				\item \textbf{Coursework:} {Software Developing Processes \& Methodologies; Software Configuration Management; Software Developing Processes \& Methodologies; Software Configuration Management; }
			\end{itemize}
		\end{cvsubsection}
	\end{cvsection}

	%Print the content of technical Experience%
	\begin{cvsection}{Experiência Técnica}
		\begin{cvsubsection}{Projetos:}{}{}
			\begin{itemize}
	    	\item \textbf{MaisMorra (2017):} Jogo Publicado, em versão Alfa, atualmente em evolução. Permitiu conhecimento de marketing, business e Game Design [Unity, C\#].
			\item \textbf{CrossRoads - UnB (2017):} Game Design e construção da Engine. Gerenciando, interpretando ideias e liderando um time multi-funcional. [ C++, SDL 2 ]
			\item \textbf{Iniciação Científica (2017):} Modeling and analyzing concepts of Hofstadter's recursive functions using genetic algorithms. Orienter: Dr. Ronni Amorim.
	      	\item \textbf{Portal Solidaridad (2016):} Plataforma de Comércio eletrônico e Informação. Objetiva permitir novas fontes de renda e dar visibilidade às entidades beneficentes.
			\end{itemize}
		\end{cvsubsection}
	\end{cvsection}

    %Print the content of languages and technologies%
	\begin{cvsection}{Technologias e Linguagens}
		\begin{cvsubsection}{}{}{}
			\begin{itemize}
			\item \textbf{Back End:} Python, Java, C, C++, JavaScript(Node.JS), PHP, C\#, Shell;
			\item \textbf{Design Gráfico:} Unity, Gimp, Inkscape, Catia, Photoshop, AutoCAD, Blender;
	      	\item \textbf{Front End:} JavaScript (Angular, JQuery, Vue.JS), HTML \& CSS, Jenkyll, WordPress;
	      	\item \textbf{Software Technologies:} Git, Docker, Gulp, Travis CI, Vagrant;
			\end{itemize}
		\end{cvsubsection}
	\end{cvsection}
\end{cvsection}
\end{document}
